% Options for packages loaded elsewhere
\PassOptionsToPackage{unicode}{hyperref}
\PassOptionsToPackage{hyphens}{url}
%
\documentclass[
  ignorenonframetext,
]{beamer}
\usepackage{pgfpages}
\setbeamertemplate{caption}[numbered]
\setbeamertemplate{caption label separator}{: }
\setbeamercolor{caption name}{fg=normal text.fg}
\beamertemplatenavigationsymbolsempty
% Prevent slide breaks in the middle of a paragraph
\widowpenalties 1 10000
\raggedbottom
\setbeamertemplate{part page}{
  \centering
  \begin{beamercolorbox}[sep=16pt,center]{part title}
    \usebeamerfont{part title}\insertpart\par
  \end{beamercolorbox}
}
\setbeamertemplate{section page}{
  \centering
  \begin{beamercolorbox}[sep=12pt,center]{part title}
    \usebeamerfont{section title}\insertsection\par
  \end{beamercolorbox}
}
\setbeamertemplate{subsection page}{
  \centering
  \begin{beamercolorbox}[sep=8pt,center]{part title}
    \usebeamerfont{subsection title}\insertsubsection\par
  \end{beamercolorbox}
}
\AtBeginPart{
  \frame{\partpage}
}
\AtBeginSection{
  \ifbibliography
  \else
    \frame{\sectionpage}
  \fi
}
\AtBeginSubsection{
  \frame{\subsectionpage}
}
\usepackage{lmodern}
\usepackage{amssymb,amsmath}
\usepackage{ifxetex,ifluatex}
\ifnum 0\ifxetex 1\fi\ifluatex 1\fi=0 % if pdftex
  \usepackage[T1]{fontenc}
  \usepackage[utf8]{inputenc}
  \usepackage{textcomp} % provide euro and other symbols
\else % if luatex or xetex
  \usepackage{unicode-math}
  \defaultfontfeatures{Scale=MatchLowercase}
  \defaultfontfeatures[\rmfamily]{Ligatures=TeX,Scale=1}
\fi
% Use upquote if available, for straight quotes in verbatim environments
\IfFileExists{upquote.sty}{\usepackage{upquote}}{}
\IfFileExists{microtype.sty}{% use microtype if available
  \usepackage[]{microtype}
  \UseMicrotypeSet[protrusion]{basicmath} % disable protrusion for tt fonts
}{}
\makeatletter
\@ifundefined{KOMAClassName}{% if non-KOMA class
  \IfFileExists{parskip.sty}{%
    \usepackage{parskip}
  }{% else
    \setlength{\parindent}{0pt}
    \setlength{\parskip}{6pt plus 2pt minus 1pt}}
}{% if KOMA class
  \KOMAoptions{parskip=half}}
\makeatother
\usepackage{xcolor}
\IfFileExists{xurl.sty}{\usepackage{xurl}}{} % add URL line breaks if available
\IfFileExists{bookmark.sty}{\usepackage{bookmark}}{\usepackage{hyperref}}
\hypersetup{
  pdftitle={¿Cómo iniciar en la academia?},
  pdfauthor={Paola Villa Paro},
  hidelinks,
  pdfcreator={LaTeX via pandoc}}
\urlstyle{same} % disable monospaced font for URLs
\newif\ifbibliography
\setlength{\emergencystretch}{3em} % prevent overfull lines
\providecommand{\tightlist}{%
  \setlength{\itemsep}{0pt}\setlength{\parskip}{0pt}}
\setcounter{secnumdepth}{-\maxdimen} % remove section numbering

\title{¿Cómo iniciar en la academia?}
\subtitle{RAship como medio de mobilidad académica}
\author{Paola Villa Paro}
\date{Princeton University \textbar{} \href{https://pvillaparo.github.io}{Bio}}

\begin{document}
\frame{\titlepage}

\begin{frame}{Table of contents}
\protect\hypertarget{table-of-contents}{}

\begin{enumerate}
\item
  \protect\hyperlink{experiencia}{Mi experiencia postulando al Pre-Doc}
\item
  \protect\hyperlink{comose}{¿Cómo sé si me gusta la academia?}
\item
  \protect\hyperlink{comoapp}{¿Cómo Aplicar?}
\item
  \protect\hyperlink{fuentes}{Fuentes}
\end{enumerate}

\end{frame}

\begin{frame}{Mi Background}
\protect\hypertarget{mi-background}{}

\begin{itemize}
\item
  Economista.
\item
  Interés en temas de mobilidad social, desigualdad, y poblaciones
  vulnerables.
\item
  Experiencia como RA en IPA y CIUP.
\item
  Actualmente en Princeton University como Pre-doctoral fellow.

  \begin{itemize}
  \tightlist
  \item
    Donde veo temas de mobilidad social, género, y economía política
  \end{itemize}
\end{itemize}

\end{frame}

\begin{frame}{Lugares internacionales con programas extensos para
RAships}
\protect\hypertarget{lugares-internacionales-con-programas-extensos-para-raships}{}

\end{frame}

\begin{frame}{Actividades comúnes en estos puestos}
\protect\hypertarget{actividades-comuxfanes-en-estos-puestos}{}

\end{frame}

\begin{frame}{Softwares comúnmente usados}
\protect\hypertarget{softwares-comuxfanmente-usados}{}

\end{frame}

\begin{frame}{Distribución del tiempo de un RA}
\protect\hypertarget{distribuciuxf3n-del-tiempo-de-un-ra}{}

\end{frame}

\begin{frame}{Pequeña diferencia entre un Predoctoral fellowship y un
RAship}
\protect\hypertarget{pequeuxf1a-diferencia-entre-un-predoctoral-fellowship-y-un-raship}{}

\begin{itemize}
\tightlist
\item
  Son asistentes de investigación graduados del pregrado con énfasis en
  seguir un PhD.
\item
  Las herramientas dentro del programa te permiten prepararte para
  aplicar
\end{itemize}

\end{frame}

\begin{frame}{Requisitos para postular}
\protect\hypertarget{requisitos-para-postular}{}

\end{frame}

\begin{frame}

class: inverse, center, middle name: fuentes

\end{frame}

\begin{frame}{Fuentes}
\protect\hypertarget{fuentes}{}

\end{frame}

\begin{frame}{Fuentes:}
\protect\hypertarget{fuentes-1}{}

\begin{itemize}
\item
  Econ RA Guide: \url{https://raguide.github.io}
\item
  Survey of Pre-Doctoral Research Experiences in Economics:
  \url{https://z-y-huang.github.io/predoc_survey/slides.pdf}
\end{itemize}

\end{frame}

\begin{frame}

class: inverse, center, middle name: gracias

\end{frame}

\begin{frame}{Gracias!}
\protect\hypertarget{gracias}{}

Contacto: Paola Villa Paro ,
\href{mailto:pparo@princeton.edu}{\nolinkurl{pparo@princeton.edu}}

\end{frame}

\end{document}
