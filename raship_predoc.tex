% Options for packages loaded elsewhere
\PassOptionsToPackage{unicode}{hyperref}
\PassOptionsToPackage{hyphens}{url}
%
\documentclass[
]{article}
\usepackage{lmodern}
\usepackage{amssymb,amsmath}
\usepackage{ifxetex,ifluatex}
\ifnum 0\ifxetex 1\fi\ifluatex 1\fi=0 % if pdftex
  \usepackage[T1]{fontenc}
  \usepackage[utf8]{inputenc}
  \usepackage{textcomp} % provide euro and other symbols
\else % if luatex or xetex
  \usepackage{unicode-math}
  \defaultfontfeatures{Scale=MatchLowercase}
  \defaultfontfeatures[\rmfamily]{Ligatures=TeX,Scale=1}
\fi
% Use upquote if available, for straight quotes in verbatim environments
\IfFileExists{upquote.sty}{\usepackage{upquote}}{}
\IfFileExists{microtype.sty}{% use microtype if available
  \usepackage[]{microtype}
  \UseMicrotypeSet[protrusion]{basicmath} % disable protrusion for tt fonts
}{}
\makeatletter
\@ifundefined{KOMAClassName}{% if non-KOMA class
  \IfFileExists{parskip.sty}{%
    \usepackage{parskip}
  }{% else
    \setlength{\parindent}{0pt}
    \setlength{\parskip}{6pt plus 2pt minus 1pt}}
}{% if KOMA class
  \KOMAoptions{parskip=half}}
\makeatother
\usepackage{xcolor}
\IfFileExists{xurl.sty}{\usepackage{xurl}}{} % add URL line breaks if available
\IfFileExists{bookmark.sty}{\usepackage{bookmark}}{\usepackage{hyperref}}
\hypersetup{
  pdftitle={¿Cómo iniciar en la academia?},
  pdfauthor={Paola Villa Paro},
  hidelinks,
  pdfcreator={LaTeX via pandoc}}
\urlstyle{same} % disable monospaced font for URLs
\usepackage[margin=1in]{geometry}
\usepackage{graphicx,grffile}
\makeatletter
\def\maxwidth{\ifdim\Gin@nat@width>\linewidth\linewidth\else\Gin@nat@width\fi}
\def\maxheight{\ifdim\Gin@nat@height>\textheight\textheight\else\Gin@nat@height\fi}
\makeatother
% Scale images if necessary, so that they will not overflow the page
% margins by default, and it is still possible to overwrite the defaults
% using explicit options in \includegraphics[width, height, ...]{}
\setkeys{Gin}{width=\maxwidth,height=\maxheight,keepaspectratio}
% Set default figure placement to htbp
\makeatletter
\def\fps@figure{htbp}
\makeatother
\setlength{\emergencystretch}{3em} % prevent overfull lines
\providecommand{\tightlist}{%
  \setlength{\itemsep}{0pt}\setlength{\parskip}{0pt}}
\setcounter{secnumdepth}{-\maxdimen} % remove section numbering

\title{¿Cómo iniciar en la academia?}
\usepackage{etoolbox}
\makeatletter
\providecommand{\subtitle}[1]{% add subtitle to \maketitle
  \apptocmd{\@title}{\par {\large #1 \par}}{}{}
}
\makeatother
\subtitle{RAship como medio de mobilidad académica}
\author{Paola Villa Paro}
\date{Princeton University \textbar{} \href{https://pvillaparo.github.io}{Bio}}

\begin{document}
\maketitle

\hypertarget{table-of-contents}{%
\section{Table of contents}\label{table-of-contents}}

\begin{enumerate}
\def\labelenumi{\arabic{enumi}.}
\item
  \protect\hyperlink{experiencia}{Mi experiencia postulando al Pre-Doc}
\item
  \protect\hyperlink{comose}{¿Cómo sé si me gusta la academia?}
\item
  \protect\hyperlink{comoapp}{¿Cómo Aplicar?}
\item
  \protect\hyperlink{fuentes}{Fuentes}
\end{enumerate}

\hypertarget{mi-background}{%
\section{Mi Background}\label{mi-background}}

\begin{itemize}
\item
  Economista.
\item
  Interés en temas de mobilidad social, desigualdad, y poblaciones
  vulnerables.
\item
  Experiencia como RA en IPA y CIUP.
\item
  Actualmente en Princeton University como Pre-doctoral fellow.

  \begin{itemize}
  \tightlist
  \item
    Donde veo temas de mobilidad social, género, y economía política
  \end{itemize}
\end{itemize}

\hypertarget{lugares-internacionales-con-programas-extensos-para-raships}{%
\section{Lugares internacionales con programas extensos para
RAships}\label{lugares-internacionales-con-programas-extensos-para-raships}}

\begin{center}\rule{0.5\linewidth}{0.5pt}\end{center}

\hypertarget{actividades-comuxfanes-en-estos-puestos}{%
\section{Actividades comúnes en estos
puestos}\label{actividades-comuxfanes-en-estos-puestos}}

\begin{center}\rule{0.5\linewidth}{0.5pt}\end{center}

\hypertarget{softwares-comuxfanmente-usados}{%
\section{Softwares comúnmente
usados}\label{softwares-comuxfanmente-usados}}

\begin{center}\rule{0.5\linewidth}{0.5pt}\end{center}

\hypertarget{distribuciuxf3n-del-tiempo-de-un-ra}{%
\section{Distribución del tiempo de un
RA}\label{distribuciuxf3n-del-tiempo-de-un-ra}}

\begin{center}\rule{0.5\linewidth}{0.5pt}\end{center}

\hypertarget{pequeuxf1a-diferencia-entre-un-predoctoral-fellowship-y-un-raship}{%
\section{Pequeña diferencia entre un Predoctoral fellowship y un
RAship}\label{pequeuxf1a-diferencia-entre-un-predoctoral-fellowship-y-un-raship}}

\begin{itemize}
\tightlist
\item
  Son asistentes de investigación graduados del pregrado con énfasis en
  seguir un PhD.
\item
  Las herramientas dentro del programa te permiten prepararte para
  aplicar
\end{itemize}

\hypertarget{requisitos-para-postular}{%
\section{Requisitos para postular}\label{requisitos-para-postular}}

\begin{center}\rule{0.5\linewidth}{0.5pt}\end{center}

class: inverse, center, middle name: fuentes

\hypertarget{fuentes}{%
\section{Fuentes}\label{fuentes}}

\begin{center}\rule{0.5\linewidth}{0.5pt}\end{center}

\hypertarget{fuentes-1}{%
\section{Fuentes:}\label{fuentes-1}}

\begin{itemize}
\item
  Econ RA Guide: \url{https://raguide.github.io}
\item
  Survey of Pre-Doctoral Research Experiences in Economics:
  \url{https://z-y-huang.github.io/predoc_survey/slides.pdf}
\end{itemize}

\begin{center}\rule{0.5\linewidth}{0.5pt}\end{center}

class: inverse, center, middle name: gracias

\hypertarget{gracias}{%
\section{Gracias!}\label{gracias}}

Contacto: Paola Villa Paro ,
\href{mailto:pparo@princeton.edu}{\nolinkurl{pparo@princeton.edu}}

\end{document}
